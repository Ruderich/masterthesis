\documentclass[12pt]{article}
\usepackage[english]{babel}
\usepackage{amsmath}
\usepackage{amssymb}
\usepackage{amsmath, amssymb}
\usepackage{bm}

\usepackage[colorinlistoftodos,prependcaption,textsize=tiny]{todonotes} % great for keeping track of todos
\usepackage{xargs}                      % Use more than one optional parameter in a new commands
\usepackage{xcolor}  % Coloured text etc.
\newcommandx{\info}[2][1=]{\todo[linecolor=green,backgroundcolor=green!25,bordercolor=green,#1]{\textbf{Info:} #2}}
\newcommandx{\toadd}[2][1=]{\todo[linecolor=blue,backgroundcolor=blue!25,bordercolor=blue,#1]{\textbf{To-add:} #2}}
\newcommandx{\change}[2][1=]{\todo[linecolor=yellow,backgroundcolor=yellow!25,bordercolor=yellow,#1]{\textbf{Change:} #2}}
\newcommandx{\error}[2][1=]{\todo[linecolor=red,backgroundcolor=red!25,bordercolor=red,#1]{#2}}

\title{Application of Gaussian Process regression to the Estimation of time-in-therapeutic-range (TTR) and prediction of the international normalized ratio (INR) in patients under therapy with Vitamin K antagonists}
\author{Franz Ruderich}

\begin{document}
	\maketitle
	
	\clearpage
	\listoftodos
	
	\vspace{2\baselineskip}

    \info[inline]{I would recommend splitting chapters into separate file and 
    link them in the main file via `input' command}
    \info[inline]{Keep track of literature and citations right away! (Formula xyz, cf [?] p. xyz ...)}
	
	\clearpage
	
	\section{Background}
	\toadd[inline]{describe your data, include a panel of `typical' patients, maybe also
	some less `nice'  ones where we could run into problems}
	
	\section{Existing work}
	\toadd[inline]{describe piecewise linear model and show by example of one or two patients
	that the model is sometimes implausible}
	
	\section{Data and methods}
	
	\subsection{Gaussian Processes for Regression}
	For noisy measurements of (laboratory) values at certain time points it is difficult to fix
	what the best estimates of values between the time points or at future time points are.
	The assumption of an underlying linear function makes it easy to fit a straight line by least-squares method. But this is often a oversimplification of the problem. The model will provide poor predictions, if the relationship between  dependent and independent variable can not be approximated by a linear function.
	Even a polynomial regression models the changes of laboratory measurements  over time only unsatisfactorily with the problem of over-fitting
	Regression with Gaussian processes is a smarter method to model the relationship between successive measured values. 
	A Gaussian Process does not assume some specific model underlying the data but calculates the relationship between single data points.
	In principle are Gaussian Processes a generalization of multivariate Gaussian distributions toward infinite dimensionality.
	The $n$ observations of a (laboratory) data set $y=\{y_{1},y_{2}, \ldots ,y_{n}\}$ measured at the time points $t=\{t_{1},t_{2}, \ldots ,t_{n}\}$ could be imagined as a sample of a multivariate, n-variate Gaussian distribution.
	
	A Gaussian process of a real process $f(t)$ is characterized by a mean function $m(t)$ and covariance function $k(t,t')$:
	
	\begin{displaymath}
	m(t) = E[f(t)]\\
	\end{displaymath}
	\begin{displaymath}
	k(t,t') = E[(f(t)-m(t)) (f(t')-m(t'))]
	\end{displaymath}   
	
	In the case of repeated laboratory measurements the function values $f(t)$ represent the laboratory value measured at the time point t.
	   
	A typical covariance function is the squared exponential covariance function:
	
	\begin{displaymath}
	cov( f(t_{p} ), f(t_{q} )) = 
	k( t_{p} , t_{q} ) = 
	e^{-\frac{(t_{p}-t_{q})^2}{2*l^2}}
	\end{displaymath}
	
	
    \subsection{Estimation of hyperparameters}\change{Better use proper multivariate formulation right away! Formulate in terms of model used}
	Maximum likelihood estimator: 
	\begin{itemize}
		\item unbiased: $E(\hat{\theta}) = \theta$ 
		\item consistent: $\hat{\theta} \rightarrow \theta$, as $n \rightarrow \infty$
		\item efficient: small $SE(\hat{\theta})$, as $n \rightarrow \infty$
		\item asymptotically normal: $\frac{(\hat{\theta}-\theta)}{SE(\hat{\theta})} \sim N(0,1)$
	\end{itemize}
	\subsubsection{Accuracy of Maximum Likelihood Estimation - The Fisher Information}\change{v.s. make multivariate and make it concrete for the model we actually use}
	The estimation of parameters $\bm{\theta}$ by maximizing the log-likelihood function means to find local maxima of this function. 
	The first derivation of the log-likelihood function, the score, is equated to 0:
	
	\begin{displaymath}
	\frac{\partial \ell (\theta;x)}{\partial \theta}=\frac{\partial \log p(x;\theta)}{\partial \theta} = 0
	\end{displaymath}
	
	The maxima are found by solving this equation. 
	But it is necessary to know how accurate this estiamtion is. This information is given by calculating the curvature of the likelihood function around the maxima. A sharp curvature around the maximum means a high certainty, while a flat course gives a signal for a quite uncertain estimation.
	A measure for the curvature of the score function is the variance of the score, called Fisher Information $I(\theta)$. The variance of the score is calculated by this formula:
	
	\begin{displaymath}
	I(\theta)=
	V[\frac{\partial \ell (\theta,x)}{\partial \theta}]=
	E[(\frac{\partial}{\partial\theta}\ell(\theta;x))^{2}]=	
	-E[\frac{\partial^{2} \ell (\theta;x)}{\partial \theta^{2}}]
	\end{displaymath}
	
	The MLE is asymptotically normal distributed, so a asymptotic normal approximation and calculation of a confidence interval is possible:
	
	\begin{displaymath}
	\hat{\theta} \pm 1.96 \frac{1}{\sqrt{I(\hat{\theta})}} 
	\end{displaymath}
	
	\toadd[inline]{if you add the ansatz for the computation of the Fisher 
    	information in proper formulas (define your model in properly in 
    	preceding sections) I can check it ;)}
	
	\subsection{Fisher Information Matrix for Hyperparameters of the Gaussian Process}
	As mentioned above a Gaussian Process can be seen as a multivariate Gaussian distribution:
	
	\begin{displaymath}
	GP = f_{*}|X,y,X_{*} \sim N(\bar{f}_{*},cov(f_{*}))
	\end{displaymath}
	where 
	\begin{displaymath}
		\bar{f}_{*} \triangleq \mathbb{E}[f_{*}|X,y,X_{*}] = K(X_{*},X)[K(X,X) +\sigma^{2}_{n}I]^{-1}y
	\end{displaymath}
	and
	\begin{displaymath}
		\Sigma = cov(f_{*}) = K(X_{*},X_{*})-K(X_{*},X)[K(X,X) + \sigma^{2}_{n}I]^{-1}K(X,X_{*})
	\end{displaymath}
	
	The mean-vector $\bar{f}_{*}$ and the covariance matrix $cov(f_{x})$ depend on the parameter vector $\theta = [\sigma, mean, l]$
	
	The Fisher Information Matrix for the parameters is 3x3 Matrix with the elements
	
	\begin{displaymath}
	I_{m,n} = \frac{\partial \bar{f_{*}}^{T}}{\partial \theta_{m}}
				\Sigma^{-1}
				\frac{\partial \bar{f_{*}}}{\partial \theta_{n}}+
				\frac{1}{2}
				tr(\Sigma^{-1} \frac{\partial \Sigma}{\partial \theta_{m}} \Sigma^{-1} \frac{\partial \Sigma}{\partial \theta_{n}})
	\end{displaymath}
	
	For abbreviation and better readability: 
	\begin{align*}
	& A \doteq K(X_{*},X) \\
	& B \doteq [K(X,X) +\sigma^{2}_{n}I]
	\end{align*}
	
	Using the derivation rules for matrices:
	\begin{displaymath}
	\frac{\partial}{\partial \theta}(A(\theta)B(\theta)) = (\frac{\partial}{\partial \theta}A(\theta)) B(\theta) + A(\theta)(\frac{\partial}{\partial \theta}B(\theta))
	\end{displaymath}
	
	\begin{displaymath}
	\frac{\partial}{\partial \theta}(A(\theta)^{-1}) = -A(\theta)^{-1} (\frac{\partial}{\partial \theta} A(\theta)) A(\theta)^{-1}
	\end{displaymath}

	\begin{align*}
	& \frac{\partial \bar{f_{*}}}{\partial \theta_{m}} = \frac{\partial}{\partial \theta_{m}}(K(X_{*},X)[K(X,X) +\sigma^{2}_{n}I]^{-1}y) = \\
	& \frac{\partial}{\partial \theta_{m}} (AB^{-1})y = \\
	& (\frac{\partial}{\partial \theta_{m}} AB^{-1} + A \frac{\partial}{\partial \theta_{m}}B^{-1})y = \\ 
	& (\frac{\partial}{\partial \theta_{m}} AB^{-1} + A (-B^{-1} \frac{\partial}{\partial \theta_{m}}B B^{-1}))y
	\end{align*}
	
	\begin{align*}
	& \frac{\partial}{\partial \theta_{m}}\Sigma = \frac{\partial}{\partial \theta_{m}} (K(X_{*},X_{*})-K(X_{*},X)[K(X,X) + \sigma^{2}_{n}I]^{-1}K(X,X_{*})) = \\
	& \frac{\partial}{\partial \theta_{m}} (K(X_{*},X_{*})-(AB^{-1})K(X,X_{*}))= \\
	& \frac{\partial}{\partial \theta_{m}} (K(X_{*},X_{*}) - [(\frac{\partial}{\partial \theta_{m}}AB^{-1} + A (-B^{-1} \frac{\partial}{\partial \theta_{m}}B B^{-1}))K(X,X_{*}) + (AB^{-1}) \frac{\partial}{\partial \theta} K(X,X_{*})] 
	\end{align*}
\end{document}